%\chapterimage{chapter_head_void.pdf} % Chapter heading image

\chapter{Text Chapter}


\section{Paragraphs of Text}\index{Paragraphs of Text}

\lipsum[1] % Dummy text

\begin{informationbox}{Título A}
asdasda
\lipsum[1] 
adsdasdasd
\end{informationbox}
\lipsum[1] % Dummy text

\PRLsep{Text}

\begin{frasebox}{Frase title A}{Autor}
asdasda
\lipsum[1] 
adsdasdasd
\end{frasebox}

\begin{frasebox}{Frase title B}{Autor}
asdasda
\lipsum[1] 
adsdasdasd
\end{frasebox}

\lipsum[1] % Dummy text
\begin{citando}
asdasda
\lipsum[1] 
adsdasdasd
\end{citando}

%------------------------------------------------

\section{Citation}\index{Citation}

This statement requires citation \cite{book_key}; this one is more specific \cite[122]{article_key}.


\lipsum[1-2] % Dummy text

\begin{informationbox}{Título B}
asdasda
\lipsum[1] 
adsdasdasd
\end{informationbox}


\lipsum[1-2] % Dummy text
\begin{elaborationbox}{Título C}
asdasda
\lipsum[1] 
adsdasdasd
\end{elaborationbox}

%------------------------------------------------

\section{Lists}\index{Lists}

Lists are useful to present information in a concise and/or ordered way\footnote{Footnote example...}.

\subsection{Numbered List}\index{Lists!Numbered List}

\begin{enumerate}
\item The first item
\item The second item
\item The third item
\end{enumerate}

\subsection{Bullet Points}\index{Lists!Bullet Points}

\begin{itemize}
\item The first item
\item The second item
\item The third item
\end{itemize}

\subsection{Descriptions and Definitions}\index{Lists!Descriptions and Definitions}

\begin{description}
\item[Name] Description
\item[Word] Definition
\item[Comment] Elaboration
\end{description}

