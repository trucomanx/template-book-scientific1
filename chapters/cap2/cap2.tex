\chapter{In-text Elements}

\section{Boxs}\index{Boxs}

This section shows a set of enviroments that pertain to the family of theorem box.
These can be the 
\textbf{theorem}, 
\textbf{definition}, 
\textbf{exercise}, 
\textbf{example} and 
\textbf{proofraw} enviroment 

%-------------------------------------------------------------------------------
\subsection{Theorems}\index{Theorems!Several Equations}

We can reference the Theorem \ref{theo:A} and \ref{theo:E}.
This is an example of theorem.
\begin{highlightbox}
\begin{verbatim}
\begin{theorem}
\label{theo:A}
\lipsum[1][1-3]
\begin{equation}
x^2+e^{-\frac{x^2}{2}}=1
\end{equation}
\end{theorem}
\end{verbatim}
\end{highlightbox}

The last code generates the next theorem.
\begin{theorem}
\label{theo:A}
\lipsum[1][1-3]
\begin{equation}
x^2+e^{-\frac{x^2}{2}}=1
\end{equation}
\end{theorem}



\begin{highlightbox}
\begin{verbatim}
\begin{theorem}[Theorem name]
\label{theo:E}
\lipsum[1][1-3]
\begin{equation}
x^2+e^{-\frac{x^2}{2}}=1
\end{equation}
\end{theorem}
\end{verbatim}
\end{highlightbox}
\begin{theorem}[Theorem name]
\label{theo:E}
\lipsum[1][1-3]
\begin{equation}
x^2+e^{-\frac{x^2}{2}}=1
\end{equation}
\end{theorem}

%-------------------------------------------------------------------------------
\subsection{Lemma}\index{Lemma!Several Equations}

We can reference the Lemma \ref{lemma:A} and \ref{lemma:E}.
This is an example of lemma.
\begin{highlightbox}
\begin{verbatim}
\begin{lemma}
\label{lemma:A}
\lipsum[1][1-3]
\begin{equation}
x^2+e^{-\frac{x^2}{2}}=1
\end{equation}
\end{lemma}
\end{verbatim}
\end{highlightbox}

The last code generates the next lemma.
\begin{lemma}
\label{lemma:A}
\lipsum[1][1-3]
\begin{equation}
x^2+e^{-\frac{x^2}{2}}=1
\end{equation}
\end{lemma}



\begin{highlightbox}
\begin{verbatim}
\begin{lemma}[Lemma name]
\label{lemma:E}
\lipsum[1][1-3]
\begin{equation}
x^2+e^{-\frac{x^2}{2}}=1
\end{equation}
\end{lemma}
\end{verbatim}
\end{highlightbox}
\begin{lemma}[Lemma name]
\label{lemma:E}
\lipsum[1][1-3]
\begin{equation}
x^2+e^{-\frac{x^2}{2}}=1
\end{equation}
\end{lemma}


%-------------------------------------------------------------------------------
\subsection{Proof}\index{Proof!Single Line}
This is an example of theorem proof.
\begin{highlightbox}
\begin{verbatim}
\begin{proofraw}%
[Relative to Teorema \ref{theo:A}]
\lipsum[1][1-3]
\begin{equation}
x^2+e^{-\frac{x^2}{2}}=1
\end{equation}
\end{proofraw}
\end{verbatim}
\end{highlightbox}

The last code gnerates the next proof.
\begin{proofraw}[Relative to Teorema \ref{theo:A}]
\lipsum[1][1-3]
\begin{equation}
x^2+e^{-\frac{x^2}{2}}=1
\end{equation}
\end{proofraw}

%-------------------------------------------------------------------------------
\subsection{Exercises}\index{Exercises!Single Line}


We can reference the Exercises \ref{exer:A} and \ref{exer:B}.
This is an example of exercise.
\begin{highlightbox}
\begin{verbatim}
\begin{exercise}[Exercise name A]
\label{exer:A}
\lipsum[1][1-3]
\begin{equation}
x^2+e^{-\frac{x^2}{2}}=1
\end{equation}
\end{exercise}
\end{verbatim}
\end{highlightbox}

The last code generates the next exercise.
\begin{exercise}[Exercise name A]
\label{exer:A}
\lipsum[1][1-3]
\begin{equation}
x^2+e^{-\frac{x^2}{2}}=1
\end{equation}
\end{exercise}

This is an example of exercise without title.
\begin{highlightbox}
\begin{verbatim}
\begin{exercise}
\label{exer:B}
\lipsum[1][1-3]
\begin{equation}
x^2+e^{-\frac{x^2}{2}}=1
\end{equation}
\end{exercise}
\end{verbatim}
\end{highlightbox}

The last code generates the next exercise.
\begin{exercise}
\label{exer:B}
\lipsum[1][1-3]
\begin{equation}
x^2+e^{-\frac{x^2}{2}}=1
\end{equation}
\end{exercise}


%-------------------------------------------------------------------------------
\subsection{Examples}\index{Examples}

We can reference the Examples \ref{exam:A} and \ref{exam:B}.
This is an example of example box.

\begin{highlightbox}
\begin{verbatim}
\begin{example}[Example name]
\label{exam:A}
\lipsum[1][1-3]
\begin{equation}
x^2+e^{-\frac{x^2}{2}}=1
\end{equation}
\end{example}
\end{verbatim}
\end{highlightbox}

The last code generates the next exercise.
\begin{example}[Example name]
\label{exam:A}
\lipsum[1][1-3]
\begin{equation}
x^2+e^{-\frac{x^2}{2}}=1
\end{equation}
\end{example}

This is an example of example box without title.
\begin{highlightbox}
\begin{verbatim}
\begin{example}
\label{exam:B}
\lipsum[1][1-3]
\begin{equation}
x^2+e^{-\frac{x^2}{2}}=1
\end{equation}
\end{example}
\end{verbatim}
\end{highlightbox}

The last code generates the next exercise.
\begin{example}
\label{exam:B}
\lipsum[1][1-3]
\begin{equation}
x^2+e^{-\frac{x^2}{2}}=1
\end{equation}
\end{example}
%-------------------------------------------------------------------------------
\subsection{Definitions}\index{Definitions}

Definition \ref{def:A0},

\begin{highlightbox}
\begin{verbatim}
\begin{definition}[Definition name]
\label{def:A0}
\lipsum[1][1-3]
\begin{equation}
x^2+e^{-\frac{x^2}{2}}=1
\end{equation}
\end{definition}
\end{verbatim}
\end{highlightbox}

\begin{definition}[Definition name]
\label{def:A0}
\lipsum[1][1-3]
\begin{equation}
x^2+e^{-\frac{x^2}{2}}=1
\end{equation}
\end{definition}




%-------------------------------------------------------------------------------
\subsection{Notations}\index{Notations}

Notations \ref{not:A} and \ref{not:B}
\begin{highlightbox}
\begin{verbatim}
\begin{notation}[Notation name title very large]
\label{not:A}
\lipsum[1][1-3]
\begin{equation}
x^2+e^{-\frac{x^2}{2}}=1
\end{equation}
\end{notation}
\end{verbatim}
\end{highlightbox}
\begin{notation}[Notation name title very large]
\label{not:A}
\lipsum[1][1-3]
\begin{equation}
x^2+e^{-\frac{x^2}{2}}=1
\end{equation}
\end{notation}

\begin{highlightbox}
\begin{verbatim}
\begin{notation}
\label{not:B}
\lipsum[1][1-3]
\begin{equation}
x^2+e^{-\frac{x^2}{2}}=1
\end{equation}
\end{notation}
\end{verbatim}
\end{highlightbox}
\begin{notation}
\label{not:B}
\lipsum[1][1-3]
\begin{equation}
x^2+e^{-\frac{x^2}{2}}=1
\end{equation}
\end{notation}





%------------------------------------------------

\section{Equationbox}\index{Equationbox}

\begin{highlightbox}
\begin{verbatim}
\begin{equationbox}
\begin{equation}
x^2+e^{-\frac{x^2}{2}}=1
\end{equation}
\end{equationbox}
\end{verbatim}
\end{highlightbox}
\begin{equationbox}
\begin{equation}
x^2+e^{-\frac{x^2}{2}}=1
\end{equation}
\end{equationbox}


\section{Phrasebox}\index{Phrasebox}
\begin{highlightbox}
\begin{verbatim}
\begin{phrasebox}{Frase title}{Fernando P. R.}
\lipsum[1][1-3]
\end{phrasebox}
\end{verbatim}
\end{highlightbox}
\begin{phrasebox}{Frase title}{Fernando P. R.}
\lipsum[1][1-3]
\end{phrasebox}
%------------------------------------------------

\section{Attentionbox}\index{Attentionbox}

\lipsum[1][1-3]
\begin{highlightbox}
\begin{verbatim}
\begin{attentionbox}
\lipsum[1][1-3] 
\end{attentionbox}
\end{verbatim}
\end{highlightbox}
\begin{attentionbox}
\lipsum[1][1-3] 
\end{attentionbox}


%------------------------------------------------

\section{Informationbox}\index{Informationbox}

\lipsum[1][1-3]
\begin{highlightbox}
\begin{verbatim}
\begin{informationbox}[Título A]
\lipsum[1][1-3]
\end{informationbox}
\end{verbatim}
\end{highlightbox}
\begin{informationbox}[Título A]
\lipsum[1][1-3]
\end{informationbox}

%------------------------------------------------

\section{Elaboration}\index{Elaboration}

The next code show how to use the \texttt{elaboration} enviroment with title.
\begin{highlightbox}
\begin{verbatim}
\begin{elaboration}[Title of elaboration]
\lipsum[1][1-6]
\end{elaboration}
\end{verbatim}
\end{highlightbox}
The result can be seen in the next \texttt{elaboration}.
\begin{elaboration}[Title of elaboration]
\lipsum[1][1-6]
\end{elaboration}

The next code show how to use the \texttt{elaboration} enviroment without title.
\begin{highlightbox}
\begin{verbatim}
\begin{elaboration}
\lipsum[1][1-6]
\end{elaboration}
\end{verbatim}
\end{highlightbox}
The result can be seen in the next \texttt{elaboration}.
\begin{elaboration}
\lipsum[1][1-6]
\end{elaboration}

%------------------------------------------------

\section{Notebox}\index{Notebox}
\begin{highlightbox}
\begin{verbatim}
\begin{notebox}
\lipsum[1][1-3]
\end{notebox}
\end{verbatim}
\end{highlightbox}
\begin{notebox}
\lipsum[1][1-3]
\end{notebox}

%-------------------------------------------------------------------------------
\section{Bullet journal }\index{Bullet journal}

%------------------------------------------------
\subsection{Bulletjournalitem}\index{Bulletjournalitem}

\lipsum[1][1-3]
\begin{highlightbox}
\begin{verbatim}
\begin{bulletjournalitem}
\tcbitem \lipsum[1][1-3]
\tcbitem \lipsum[1][2-4]
\tcbitem \lipsum[1][3-5]
\end{bulletjournalitem}
\end{verbatim}
\end{highlightbox}
\begin{bulletjournalitem}
\tcbitem \lipsum[1][1-3]
\tcbitem \lipsum[1][2-4]
\tcbitem \lipsum[1][3-5]
\end{bulletjournalitem}

%------------------------------------------------
\subsection{Bulletjournalpicture}\index{Bulletjournalpicture}

\lipsum[1][1-3]
\begin{highlightbox}
\begin{verbatim}
\begin{bulletjournalpicture}[black]
\tcbitem \lipsum[1][1-3]
\tcbitem \lipsum[1][2-4]
\tcbitem \lipsum[1][3-5]
\end{bulletjournalpicture}
\end{verbatim}
\end{highlightbox}
\begin{bulletjournalpicture}[black]
\tcbitem \lipsum[1][1-3]
\tcbitem \lipsum[1][2-4]
\tcbitem \lipsum[1][3-5]
\end{bulletjournalpicture}

%------------------------------------------------
\subsection{Bulletjournalarrow}\index{Bulletjournalarrow}

\lipsum[1][1-3]
\begin{highlightbox}
\begin{verbatim}
\begin{bulletjournalarrow}[black]
\tcbitem First line of text.
\tcbitem Second line of text.
\end{bulletjournalarrow}
\end{verbatim}
\end{highlightbox}
\begin{bulletjournalarrow}[black]
\tcbitem First line of text.
\tcbitem Second line of text.
\end{bulletjournalarrow}



