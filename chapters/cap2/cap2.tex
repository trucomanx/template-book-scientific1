\chapter{In-text Elements}

\section{Boxs}\index{Boxs}

This is an example of theorems. Theorema \ref{theo:A}, \ref{theo:B}, \ref{theo:C}, \ref{theo:D} e \ref{theo:E}.

%-------------------------------------------------------------------------------
\subsection{Theorems}\index{Theorems!Several Equations}


\begin{verbatim}
\begin{theorem}[Name of the theorem]
\label{theo:A}
\lipsum[1][1-3]
\begin{equation}
x^2+e^{-\frac{x^2}{2}}=1
\end{equation}
\end{theorem}
\end{verbatim}
\begin{theorem}[Name of the theorem]
\label{theo:A}
\lipsum[1][1-3]
\begin{equation}
x^2+e^{-\frac{x^2}{2}}=1
\end{equation}
\end{theorem}

\begin{verbatim}
\begin{proofraw}[Relative to Teorema \ref{theo:A}]
\lipsum[1][1-3]
\begin{equation}
x^2+e^{-\frac{x^2}{2}}=1
\end{equation}
\end{proofraw}
\end{verbatim}
\begin{proofraw}[Relative to Teorema \ref{theo:A}]
\lipsum[1][1-3]
\begin{equation}
x^2+e^{-\frac{x^2}{2}}=1
\end{equation}
\end{proofraw}

\subsection{Exercises}\index{Exercises!Single Line}
\lipsum[1][1-3]

\begin{verbatim}
\begin{exercise}[Exercise name]
\lipsum[1][1-3]
\begin{equation}
x^2+e^{-\frac{x^2}{2}}=1
\end{equation}
\end{exercise}
\end{verbatim}
\begin{exercise}[Exercise name]
\lipsum[1][1-3]
\begin{equation}
x^2+e^{-\frac{x^2}{2}}=1
\end{equation}
\end{exercise}

\begin{verbatim}
\begin{exercise}[Exercise name A0]
\label{ex:A0}
\lipsum[1][1-3]
\begin{equation}
x^2+e^{-\frac{x^2}{2}}=1
\end{equation}
\end{exercise}
\end{verbatim}
\begin{exercise}[Exercise name A0]
\label{ex:A0}
\lipsum[1][1-3]
\begin{equation}
x^2+e^{-\frac{x^2}{2}}=1
\end{equation}
\end{exercise}

\begin{verbatim}
\begin{theorem}
\label{theo:B}
\lipsum[1][1-3]
\end{theorem}
\end{verbatim}
\begin{theorem}
\label{theo:B}
\lipsum[1][1-3]
\end{theorem}

%-------------------------------------------------------------------------------
\section{Examples}\index{Examples}
\lipsum[1][1-3]
See Example \ref{ex:A0}

\begin{verbatim}
\begin{example}[Example name]
\label{ex:A0:b}
\lipsum[1][1-3]
\begin{equation}
x^2+e^{-\frac{x^2}{2}}=1
\end{equation}
\end{example}
\end{verbatim}
\begin{example}[Example name]
\label{ex:A0:b}
\lipsum[1][1-3]
\begin{equation}
x^2+e^{-\frac{x^2}{2}}=1
\end{equation}
\end{example}

\begin{verbatim}
\begin{theorem}
\label{theo:C}
\lipsum[1][1-3]
\end{theorem}
\end{verbatim}
\begin{theorem}
\label{theo:C}
\lipsum[1][1-3]
\end{theorem}

\begin{verbatim}
\begin{theorem}
\label{theo:D}
\lipsum[1][1-3]
\end{theorem}
\end{verbatim}
\begin{theorem}
\label{theo:D}
\lipsum[1][1-3]
\end{theorem}


%-------------------------------------------------------------------------------
\section{Definitions}\index{Definitions}

Definition \ref{def:A0},
\begin{verbatim}
\begin{definition}[Definition name]
\label{def:A0}
\lipsum[1][1-3]
\begin{equation}
x^2+e^{-\frac{x^2}{2}}=1
\end{equation}
\end{definition}
\end{verbatim}
\begin{definition}[Definition name]
\label{def:A0}
\lipsum[1][1-3]
\begin{equation}
x^2+e^{-\frac{x^2}{2}}=1
\end{equation}
\end{definition}

\begin{verbatim}
\begin{theorem}[Theorem name]
\label{theo:E}
\lipsum[1][1-3]
\begin{equation}
x^2+e^{-\frac{x^2}{2}}=1
\end{equation}
\end{theorem}
\end{verbatim}
\begin{theorem}[Theorem name]
\label{theo:E}
\lipsum[1][1-3]
\begin{equation}
x^2+e^{-\frac{x^2}{2}}=1
\end{equation}
\end{theorem}

\begin{verbatim}
\begin{exercise}[Exercise name A]
\label{exer:A}
\lipsum[1][1-3]
\begin{equation}
x^2+e^{-\frac{x^2}{2}}=1
\end{equation}
\end{exercise}
\end{verbatim}
\begin{exercise}[Exercise name A]
\label{exer:A}
\lipsum[1][1-3]
\begin{equation}
x^2+e^{-\frac{x^2}{2}}=1
\end{equation}
\end{exercise}

\begin{verbatim}
\begin{exercise}[Exercise name]
\lipsum[1][1-3]
\begin{equation}
x^2+e^{-\frac{x^2}{2}}=1
\end{equation}
\end{exercise}
\end{verbatim}
\begin{exercise}[Exercise name]
\lipsum[1][1-3]
\begin{equation}
x^2+e^{-\frac{x^2}{2}}=1
\end{equation}
\end{exercise}

\begin{verbatim}
\begin{exercise}[Exercise name B]
\label{exer:B}
\lipsum[1][1-3]
\begin{equation}
x^2+e^{-\frac{x^2}{2}}=1
\end{equation}
\end{exercise}
\end{verbatim}
\begin{exercise}[Exercise name B]
\label{exer:B}
\lipsum[1][1-3]
\begin{equation}
x^2+e^{-\frac{x^2}{2}}=1
\end{equation}
\end{exercise}


Exer \ref{ex:A0}, \ref{exer:A} e \ref{exer:B}.

%-------------------------------------------------------------------------------
\section{Notations}\index{Notations}

Notations \ref{not:A}, \ref{not:B}, \ref{not:C} e \ref{not:D}.

\begin{verbatim}
\begin{notation}
\label{not:A}
\lipsum[1][1-3]
\begin{equation}
x^2+e^{-\frac{x^2}{2}}=1
\end{equation}
\end{notation}
\end{verbatim}
\begin{notation}
\label{not:A}
\lipsum[1][1-3]
\begin{equation}
x^2+e^{-\frac{x^2}{2}}=1
\end{equation}
\end{notation}

\begin{verbatim}
\begin{notation}[Notation name title very large]
\label{not:B}
\lipsum[1][1-3]
\begin{equation}
x^2+e^{-\frac{x^2}{2}}=1
\end{equation}
\end{notation}
\end{verbatim}
\begin{notation}[Notation name title very large]
\label{not:B}
\lipsum[1][1-3]
\begin{equation}
x^2+e^{-\frac{x^2}{2}}=1
\end{equation}
\end{notation}

\begin{verbatim}
\begin{notation}
\label{not:C}
\lipsum[1][1-3]
\begin{equation}
x^2+e^{-\frac{x^2}{2}}=1
\end{equation}
\end{notation}
\end{verbatim}
\begin{notation}
\label{not:C}
\lipsum[1][1-3]
\begin{equation}
x^2+e^{-\frac{x^2}{2}}=1
\end{equation}
\end{notation}

\begin{verbatim}
\begin{notation}[Notation title]
\label{not:D}
\lipsum[1][1-3]
\end{notation}
\end{verbatim}
\begin{notation}[Notation title]
\label{not:D}
\lipsum[1][1-3]
\end{notation}

%------------------------------------------------

\section{Equationbox}\index{Equationbox}

\begin{verbatim}
\begin{equationbox}
\begin{equation}
x^2+e^{-\frac{x^2}{2}}=1
\end{equation}
\end{equationbox}
\end{verbatim}
\begin{equationbox}
\begin{equation}
x^2+e^{-\frac{x^2}{2}}=1
\end{equation}
\end{equationbox}


\section{Phrasebox}\index{Phrasebox}

\begin{verbatim}
\begin{phrasebox}{Frase title}{Fernando P. R.}
\lipsum[1][1-3]
\end{phrasebox}
\end{verbatim}
\begin{phrasebox}{Frase title}{Fernando P. R.}
\lipsum[1][1-3]
\end{phrasebox}
%------------------------------------------------

\section{Attentionbox}\index{Attentionbox}

\lipsum[1][1-3]

\begin{verbatim}
\begin{attentionbox}
\lipsum[1][1-3] 
\end{attentionbox}
\end{verbatim}
\begin{attentionbox}
\lipsum[1][1-3] 
\end{attentionbox}


%------------------------------------------------

\section{Informationbox}\index{Informationbox}

\lipsum[1][1-3]

\begin{verbatim}
\begin{informationbox}[Título A]
\lipsum[1][1-3]
\end{informationbox}
\end{verbatim}
\begin{informationbox}[Título A]
\lipsum[1][1-3]
\end{informationbox}

%------------------------------------------------

\section{Elaborationbox}\index{Elaborationbox}

\lipsum[1][1-3]

\begin{verbatim}
\begin{elaborationbox}[Title of elaborationbox]
\lipsum[1][1-3]
\end{elaborationbox}
\end{verbatim}
\begin{elaborationbox}[Title of elaborationbox]
\lipsum[1][1-3]
\end{elaborationbox}

\lipsum[1][1-3]
\begin{verbatim}
\begin{elaborationbox}
\lipsum[1][1-3]
\end{elaborationbox}
\end{verbatim}
\begin{elaborationbox}
\lipsum[1][1-3]
\end{elaborationbox}
%------------------------------------------------

\section{Bulletjournalround}\index{Bulletjournalround}

\begin{verbatim}
\begin{bulletjournalround}[colorNotation]
\tcbitem First line of text.
\tcbitem Second line of text.
\end{bulletjournalround}
\end{verbatim}
\begin{bulletjournalround}[colorNotation]
\tcbitem First line of text.
\tcbitem Second line of text.
\end{bulletjournalround}

\lipsum[1][1-3]

\begin{verbatim}
\begin{enumerate}
\item First line of text.
\item Second line of text.
\end{enumerate}
\end{verbatim}
\begin{enumerate}
\item First line of text.
\item Second line of text.
\end{enumerate}

\lipsum[1][1-3]

\begin{verbatim}
\begin{notation}[Notation name title very large]
Text before bulletjournalround enviroment.
\begin{bulletjournalround}[colorNotation]
\tcbitem First line of text.
\tcbitem Second line of text.
\end{bulletjournalround}
Text out of bulletjournalround enviroment.
\end{notation}
\end{verbatim}
\begin{notation}[Notation name title very large]
Text before bulletjournalround enviroment.
\begin{bulletjournalround}[colorNotation]
\tcbitem First line of text.
\tcbitem Second line of text.
\end{bulletjournalround}
Text out of bulletjournalround enviroment.
\end{notation}

%------------------------------------------------

\section{Notebox}\index{Notebox}

\begin{verbatim}
\begin{notebox}
\lipsum[1][1-3]
\end{notebox}
\end{verbatim}
\begin{notebox}
\lipsum[1][1-3]
\end{notebox}


