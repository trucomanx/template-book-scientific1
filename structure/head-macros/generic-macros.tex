%-------------------------------------------------------------------------------
%	Horizontal separator line
%-------------------------------------------------------------------------------
%   \HRule{2pt}
%-------------------------------------------------------------------------------
\newcommand{\HRule}[1]{\noindent\rule{\linewidth}{#1}} % titlepage % creo el comando  \HRule regla horizontal



%-------------------------------------------------------------------------------
%	Horizontal separator line
%-------------------------------------------------------------------------------
%   \PRLsep{Text}
%-------------------------------------------------------------------------------
\newlength{\PRLlen}
\newcommand*\PRLsep[1]{\settowidth{\PRLlen}{#1}\advance\PRLlen by -\textwidth\divide\PRLlen by -2\noindent\makebox[\the\PRLlen]{\resizebox{\the\PRLlen}{1pt}{$\blacktriangleleft$}}\raisebox{-.5ex}{\textbf{#1}}\makebox[\the\PRLlen]{\resizebox{\the\PRLlen}{1pt}{$\blacktriangleright$}}\smallskip}




%-------------------------------------------------------------------------------
% Font information
%-------------------------------------------------------------------------------
%   \fprshowfont
%-------------------------------------------------------------------------------
\makeatletter
\newcommand{\fprshowfont}{codifica\c{c}\~ao: \f@encoding{},
  familia: \f@family{},
  serie: \f@series{},
  %shape: \f@shape{},
  e tamanho: \f@size{} pt
}
\makeatother

%-------------------------------------------------------------------------------
% Formato de cada entrada da tabla de contenidos dos enviroments
%-------------------------------------------------------------------------------
%   \TocEntryEnvTextFormat
%-------------------------------------------------------------------------------
\newcommand{\TocEntryEnvTextFormat}{}
%\newcommand{\TocEntryEnvTextFormat}{\addvspace{3pt}\sffamily\bfseries}



