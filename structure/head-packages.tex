
%----------------------------------------------------------------------------------------
%	XCOLOR
%----------------------------------------------------------------------------------------
\usepackage[dvipsnames*,svgnames,table]{xcolor}

%----------------------------------------------------------------------------------------
%   
%----------------------------------------------------------------------------------------
\usepackage{comment} %
\usepackage{lipsum} % Inserts dummy text
\usepackage{calc}   % For simpler calculation - used for spacing the index letter headings correctly
\usepackage{morewrites}%%! No room for a new \write.
\usepackage{booktabs} %% Required for nicer horizontal rules in tables \toprule \midrule \bottomrule
\usepackage{etoolbox} %% necesario para ifdefstring
\usepackage{fancyvrb}
\usepackage{lmodern}  % The lmodern package was used just to have access to a 80pt font size.

%----------------------------------------------------------------------------------------
%	ACENTOS
%----------------------------------------------------------------------------------------
\usepackage[utf8]{inputenc} % Required for including letters with accents
\usepackage[T1]{fontenc} % Use 8-bit encoding that has 256 glyphs

\usepackage[brazil]{babel}
%\usepackage[english]{babel} % English language/hyphenation

%----------------------------------------------------------------------------------------
%	FONTS
%----------------------------------------------------------------------------------------
\usepackage{avant} % Use the Avantgarde font for headings
%\usepackage{times} % Use the Times font for headings
\usepackage{mathptmx} % Use the Adobe Times Roman as the default text font together with math symbols from the Sym­bol, Chancery and Com­puter Modern fonts
\usepackage{microtype}% The microtype package was used to use \textls to space out the letters in "Chapter". % Slightly tweak font spacing for aesthetics



%----------------------------------------------------------------------------------------
%	PAGE HEADERS
%----------------------------------------------------------------------------------------

\usepackage{fancyhdr} % Required for header and footer configuration
\pagestyle{fancy}


%----------------------------------------------------------------------------------------
% Para 
% \begin{inparaenum}
% \item
% \end{inparaenum}
% parecido a timeze pero horizontal 
%----------------------------------------------------------------------------------------
\usepackage{paralist}
\usepackage{enumitem} % Customize lists % en ese orden primero paralist
\setlist{nolistsep} % Reduce spacing between bullet points and numbered lists


%----------------------------------------------------------------------------------------
%	VARIOUS REQUIRED PACKAGES AND CONFIGURATIONS
%----------------------------------------------------------------------------------------

\usepackage[%
paperwidth=\BookPaperWidth,
paperheight=\BookPaperHeight, 
top=\BookMarginTop,%
bottom=\BookMarginBottom,%
left=\BookMarginLeft,%
right=\BookMarginRight,%
headsep=10pt]{geometry} % Page margins

%----------------------------------------------------------------------------------------
% Graficos
%----------------------------------------------------------------------------------------
\usepackage{graphicx} % Required for including pictures
\usepackage{float}% para ter [H]
\graphicspath{{pictures/}} % Specifies the directory where pictures are stored
%\graphicspath{{pictures/}{pictures/social-network/}{pictures/chapter_head/}{pictures/rule-separator/}}
\ifdefined \EnableGrayScale
    \usepackage[GRAY]{epspdfconversion}
\fi

\usepackage{eso-pic}%%This package makes it easy to add some picture commands (background) to every page at ab-solute positions

%----------------------------------------------------------------------------------------
% Par diferentes modelos de caption.
%----------------------------------------------------------------------------------------
\usepackage{caption}
\usepackage{subcaption}
\usepackage{wrapfig}

%----------------------------------------------------------------------------------------
% ROTATION FIGURE
%----------------------------------------------------------------------------------------
% For rotating figures, tables, etc.
%  including their captions
% \begin{sidewaysfigure}[ht]
% \end{sidewaysfigure}
\usepackage[figuresleft]{rotating}


%----------------------------------------------------------------------------------------
%	Footnote
%----------------------------------------------------------------------------------------
\usepackage{scrextend} %multiple reference

%----------------------------------------------------------------------------------------
%  footnote font size
%----------------------------------------------------------------------------------------
\renewcommand{\footnotesize}{\small}

%----------------------------------------------------------------------------------------
% Page counter
%----------------------------------------------------------------------------------------
\usepackage{lastpage}

%----------------------------------------------------------------------------------------
% GLOSSARIO
%----------------------------------------------------------------------------------------
\usepackage{nomencl}

%----------------------------------------------------------------------------------------
% Para \singlespacing 
%----------------------------------------------------------------------------------------
\usepackage{setspace}

%----------------------------------------------------------------------------------------
% MUSICAL NOTATION
%----------------------------------------------------------------------------------------
\usepackage[generate,ps2eps]{abc}%%sudo apt-get install abcm2ps
\usepackage[nointegrals]{wasysym}
\usepackage{harmony} % http://linorg.usp.br/CTAN/macros/latex/contrib/harmony/harmony.pdf
\usepackage{leadsheets}%%https://ctan.math.illinois.edu/macros/latex/contrib/leadsheets/leadsheets_en.pdf


%----------------------------------------------------------------------------------------
% QR CODE
%----------------------------------------------------------------------------------------
%\usepackage{qrcode}

%----------------------------------------------------------------------------------------
%	DEFINITION OF THEOREM BOXES
%----------------------------------------------------------------------------------------
\usepackage{amsmath,amsfonts,amssymb,amsthm} % For math equations, theorems, symbols, etc

%----------------------------------------------------------------------------------------
% TIKZ
%----------------------------------------------------------------------------------------
\usepackage{tikz} % Required for drawing custom shapes
\usetikzlibrary{decorations.pathmorphing}

%----------------------------------------------------------------------------------------
% BOX - TCOLORBOX
%----------------------------------------------------------------------------------------
\usepackage[most,many]{tcolorbox}
\tcbuselibrary{skins}

%----------------------------------------------------------------------------------------
% Grafico circulo satelites
%----------------------------------------------------------------------------------------
\usepackage{smartdiagram}

%----------------------------------------------------------------------------------------
% Hacer indices
%----------------------------------------------------------------------------------------
\usepackage{makeidx} % Required to make an index
\makeindex % Tells LaTeX to create the files required for indexing


%----------------------------------------------------------------------------------------
%	BIBLIOGRAPHY AND INDEX
%----------------------------------------------------------------------------------------
\usepackage[ style=alphabetic,
             citestyle=alphabetic,
             sorting=none,
             sortcites=true,
             autopunct=true,
             babel=hyphen,
             hyperref=true,
             abbreviate=false,
             backref=true,
             backend=biber]{biblatex}
\addbibresource{bibliography/bibliography.bib} % BibTeX bibliography file
\defbibheading{bibempty}{}
%\renewcommand*{\bibfont}{\footnotesize}%%fontsize 

%----------------------------------------------------------------------------------------
%  BIBLIGRAHY font size
%----------------------------------------------------------------------------------------
\renewcommand*{\bibfont}{\footnotesize}

%----------------------------------------------------------------------------------------
% SIMILAR A ITEMIZE OU ENUMERATE
%----------------------------------------------------------------------------------------
\usepackage{tasks}

\settasks{
style=itemize, 
column-sep=5mm,
%label-format={\color{green!70!black}\large\bfseries},  
label-align=left, 
%label-offset={1mm}, 
%label-width={3mm}, 
%item-indent={1mm},
%item-format={\scshape\small}, 
%after-item-skip=1mm, 
%after-skip={10mm}
}


%----------------------------------------------------------------------------------------
% TABLE BREAKING
%----------------------------------------------------------------------------------------
\usepackage{longtable}

%----------------------------------------------------------------------------------------
% TABLE MULTIROW
%----------------------------------------------------------------------------------------
\usepackage{multirow}


%----------------------------------------------------------------------------------------
%	CONTROL PART, CHAPTER AND SECTION TITLE
%----------------------------------------------------------------------------------------
\usepackage[explicit]{titlesec} % necesario para titleformat

%----------------------------------------------------------------------------------------
%	MAIN TABLE OF CONTENTS
%----------------------------------------------------------------------------------------
\usepackage{titletoc} % Required for manipulating the table of contents

%----------------------------------------------------------------------------------------
%% OBLIGATORIAMENTE DEBE IR DESPUES DE TITLESEC Y TITLETOC
%----------------------------------------------------------------------------------------
%	HYPERLINKS IN THE DOCUMENTS 
%----------------------------------------------------------------------------------------
\usepackage{hyperref}
%% \usepackage{hyperxmp} %% da erro nao ativar- pdfcontactemail
\hypersetup{
	hidelinks,
	backref=true,
	pagebackref=true,
	hyperindex=true,
	colorlinks=true,
    citecolor=colorLink,    % color of citations
	linkcolor=black,          % color of internal links (change box color with linkbordercolor)
	%pdftoolbar=true,         % show Acrobat’s toolbar?
	%pdfmenubar=true,         % show Acrobat’s menu?
	%pdffitwindow=false,      % window fit to page when opened
	%pdfstartview={FitH},     % fits the width of the page to the window
	breaklinks=true,
	urlcolor= colorLink,
	bookmarks=true,
	bookmarksopen=false,
	pdftitle={\BookTitle - \BookSubTitle},
	pdfauthor={\BookAuthor},
	pdfsubject={\BookTitle - \BookSubTitle},
	pdfkeywords={\BookKeyWordA, \BookKeyWordB, \BookKeyWordC},
	%pdfcontactemail={\ImprimirRawEmail},
	%pdfcopyright={Creative Commons Atribuição - Não Comercial - Sem Derivações 4.0 Internacional},
	%baseurl={\BookLinkHomePage}
}


\usepackage{xurl} %melhor quebra de linha referencias

\usepackage{bookmark}%% precisa usepackage hyperref
\bookmarksetup{
open,
numbered,
addtohook={%
\ifnum\bookmarkget{level}=0 % chapter
\bookmarksetup{bold}%
\fi
\ifnum\bookmarkget{level}=-1 % part
\bookmarksetup{color=colorLink,bold}%
\fi
}
}


% Adiciona o diretório "paquetes" ao caminho de busca
\makeatletter
\def\input@path{{./packages}}
\makeatother

