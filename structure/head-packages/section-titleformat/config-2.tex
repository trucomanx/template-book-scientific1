
\usetikzlibrary{shapes,shadows,calc}
\usetikzlibrary{positioning,calc}

%\setlength{\AnchoPonta}{2\textheight}
\def \AnchoPonta{0.15\textwidth}%{0.15\textwidth}%2.5
\def \AnchoNoPonta{0.1\textwidth}


\newcommand\SecTitle[2]{%
\begin{tikzpicture}
\node (A) [rectangle,minimum width=\textwidth, minimum height=1cm,color=white,fill=colorSection, text width=\textwidth,align=left] {\hspace{\AnchoPonta}\parbox{\textwidth-\AnchoPonta}{\huge\textbf{\textsf{#1}}}};
%
% El rombo atras del numero
\fill[fill=colorSection!80] (A.north west) -- ($(A.north west)+(\AnchoNoPonta,0)$) -- ($0.5*(A.north west)-0.5*(A.north west)-(0.5*\textwidth-\AnchoPonta,0)$)--($(A.south west)+(\AnchoNoPonta,0)$)--(A.south west);
%
% Solo el numero
\node [color=white](A.north west) at ($0.5*(A.north west)-0.5*(A.north west)-(0.5*\textwidth-0.05\textwidth,0)$) {\Huge\textbf{\textsf{#2}}};

\end{tikzpicture}

}

\titleformat{\section}
{\normalfont}{}{0em}
{\SecTitle{#1}{\thesection}}

\titleformat{name=\section,numberless}
{\normalfont}{}{0em}
{\SecTitle{#1}{~}}

