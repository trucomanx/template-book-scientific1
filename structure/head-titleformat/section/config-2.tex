
\usetikzlibrary{shapes,shadows,calc}
\usetikzlibrary{positioning,calc}

\newlength{\AnchoPonta}
\setlength{\AnchoPonta}{10ex}%\def \AnchoPonta{10ex}%

\newlength{\AnchoNoPonta}
\setlength{\AnchoNoPonta}{8ex}%\def \AnchoNoPonta{8ex}%

\newlength{\SectionTitleBackHeight}
\setlength{\SectionTitleBackHeight}{5ex}%\def \SectionTitleBackHeight{3ex}

\def \SectionTitleFontColor{white}
\def \SectionTitleBackColor{colorSection}
\def \SectionTitleBackNumberColor{colorSection!70}


\newcommand\SecTitle[2]{%
\begin{tikzpicture}
\node (A) %
[%
rectangle,
minimum width=\textwidth, 
minimum height=\SectionTitleBackHeight,
text width=\textwidth-\AnchoPonta+\AnchoNoPonta,
color=\SectionTitleFontColor,
fill=\SectionTitleBackColor, 
%line width=0pt,
align=left%
]{\hspace{\AnchoPonta}\parbox{\textwidth-\AnchoPonta}{\Large\textbf{#1}}};
%
% El rombo atras del numero
\fill[fill=\SectionTitleBackNumberColor] (A.north west) -- ($(A.north west)+(\AnchoNoPonta,0)$) -- ($0.5*(A.north west)-0.5*(A.north west)-(0.5*\textwidth-\AnchoPonta,0)$)--($(A.south west)+(\AnchoNoPonta,0)$)--(A.south west);
%
% Solo el numero
\node [color=\SectionTitleFontColor](A.north west) at ($0.5*(A.north west)-0.5*(A.north west)-(0.5*\textwidth-0.05\textwidth,0)$) {\Large\textbf{#2}};
\end{tikzpicture}
}

\titleformat{\section}%command
{\normalfont}%format
{}%label
{0em}%sep
{\SecTitle{#1}{\thesection}}%before code
%[\vspace{-1ex}]%after code

\titleformat{name=\section,numberless}%command
{\normalfont}%format
{}%label
{0em}%sep
{\SecTitle{#1}{~}}%before code
%[\vspace{-1ex}]%after code


\titlespacing*{\section}
{0pt}%left
{2ex}%before
{-2ex}%after



