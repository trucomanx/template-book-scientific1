\usepackage{packages/env-bulletjournalround}

\SetEnvBulletJournalRoundFrameTickness{1pt}
\SetEnvBulletJournalRoundFrameColor{white}
\SetEnvBulletJournalRoundBackColor{white}
\SetEnvBulletJournalRoundPadding{0pt}
\SetEnvBulletJournalRoundIndent{24pt}
\SetEnvBulletJournalRoundItemFont{\bfseries\itshape}
\SetEnvBulletJournalRoundItemBackColor{colorDefault}
\SetEnvBulletJournalRoundItemFrameColor{colorDefault}
\SetEnvBulletJournalRoundItemFontColor{white}
\SetEnvBulletJournalRoundItemWidth{19pt}
\SetEnvBulletJournalRoundItemHeight{14pt}
\SetEnvBulletJournalRoundItemRound{6pt}

\begin{comment}
\RequirePackage[most]{tcolorbox}
\newcounter{TcbItemizeFancyCount}
\usetikzlibrary{shapes.geometric}

\newenvironment{tcbenumeratetext}[1][colorDefault]{%
\setcounter{TcbItemizeFancyCount}{0}
\begin{tcbitemize}[%
    raster columns=1,    
    nofloat,
    enhanced,
    raster left skip=24pt,%indent
    sharp corners=all,
    colback=white,
    colframe=white,%orange,
    toprule=0pt,
    bottomrule=0pt,
    leftrule=0pt,%4pt,
    rightrule=0pt,
    boxsep=0pt,
    left=3pt, right=3pt, %top=5pt,bottom=5pt,
    valign=center,
    overlay={% numero en rectangulo redondeado
        \coordinate (X) at ([xshift=-5.0mm,yshift=-4.0mm]frame.north west);
        \node[%
            draw,
            %ellipse,
            rounded corners=8pt,
            rotate=0,
            minimum width=24pt,
            minimum height=16pt,
            inner sep=0pt,
            color=#1,
            fill=#1,
            text=white,
            font=\itshape\bfseries
        ] at (X) 
        {\refstepcounter{TcbItemizeFancyCount}\normalsize\arabic{TcbItemizeFancyCount}};}
    ]%
}%
{\end{tcbitemize}}

\end{comment}


