\usepackage{wrapper-variable}

\NewWrapperVariableBoxRound[%
BackColor=black,
FrameColor=black,
TextFont=\normalfont\ttfamily,
TextColor=white,
ShadowColor=black!50!white,
HPaddingLength=4pt,
VPaddingLength=2pt,
MinWidthLength=0pt,
BaseLineLength=-0.5ex,
ArcLength=4pt
]{CommandBox}

%% \CommandBox{gedit some/path/to/filename.txt}


%-------------------------------------------------------------------------------
% Font information
%-------------------------------------------------------------------------------
%   \fprshowfont
%-------------------------------------------------------------------------------
\makeatletter
\newcommand{\fprshowfont}{codifica\c{c}\~ao: \f@encoding{},
  familia: \f@family{},
  serie: \f@series{},
  %shape: \f@shape{},
  e tamanho: \f@size{} pt
}
\makeatother

%-------------------------------------------------------------------------------
% Formato de cada entrada da tabla de contenidos dos enviroments
%
%  Es el código a colocar antes de cada entry de la tabla de contenidos.
%  Esta variable no existe en latex y es creada aqui, 
%  es usada por varios eviroments
% 
%  \newcommand{\TocEntryEnvTextFormat}{\addvspace{3pt}\sffamily\bfseries}
%-------------------------------------------------------------------------------
%   \TocEntryEnvTextFormat
%-------------------------------------------------------------------------------
\newcommand{\TocEntryEnvTextFormat}{}


%-------------------------------------------------------------------------------
% Esquema de colores
%-------------------------------------------------------------------------------

\makeatletter
\newcommand{\DefineBookColorScheme}[6]{
\expandafter\def\csname BookColorScheme@1\endcsname{#1}
\expandafter\def\csname BookColorScheme@2\endcsname{#2}
\expandafter\def\csname BookColorScheme@3\endcsname{#3}
\expandafter\def\csname BookColorScheme@4\endcsname{#4}
\expandafter\def\csname BookColorScheme@5\endcsname{#5}
\expandafter\def\csname BookColorScheme@6\endcsname{#6}
}
\makeatother

