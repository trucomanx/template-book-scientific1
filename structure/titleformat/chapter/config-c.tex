
\usepackage[
TitleColor=colorChapter,
TitleFontSize=22,
TitleBoxArc=3mm,
TitleBoxShadowColor=colorChapter,
PreTitleColor=colorChapter,
PreTitleFontsize=22,
PreTitleNumberBackgroundColor=colorChapter,
PreTitleNumberWidth=13mm,
PreTitleNumberHeight=16mm,
PreTitleNumberFontsize=30,
PreTitleNumberColor=white,
]
{packages/chapter-format-roundedtitle}

\begin{comment}
\titleformat{\chapter}[display]
  {\normalfont\huge\bfseries}
  {}
  {20pt}
  {%
    \begin{tcolorbox}[
      enhanced,
      rounded corners,
      shadow={1mm}{-1mm}{-2.2mm}{black!20!white},
      colback=white,
      boxrule=0mm,
      arc=3mm,
      leftrule=0pt,
      rightrule=0pt,
      fontupper=\color{colorChapter}\sffamily\bfseries\huge,
      width=\textwidth-3.2mm,  % account for shadow width.
      top=0.6cm, 
      bottom=0.6cm,
      overlay={
        \node[
          fill=colorChapter,
          line width=0cm,
          inner sep=0pt,
          text width=13mm,
          minimum height=16mm,
          align=center,
          anchor=south east,
          xshift=2.2mm, yshift=9mm,
          font=\color{white}\sffamily\bfseries\fontsize{30}{36}\selectfont
        ] 
        (chapname) at (frame.north east) {\thechapter};%el numero
        \node[color=gray,font=\large,anchor=base east,inner sep=1mm] at (chapname.base west)
        {\chaptertitlename};%% El texto capitulo  
      }
    ]
    \MakeUppercase{#1}
    \end{tcolorbox}%
  }
  
\titleformat{name=\chapter,numberless}[display]
  {\normalfont\huge\bfseries}
  {}
  {20pt}
  {%
    \begin{tcolorbox}[
      enhanced,
      rounded corners,
      shadow={1mm}{-1mm}{-2.2mm}{black!20!white},
      colback=white,
      boxrule=0mm,
      arc=3mm,
      leftrule=0pt,
      rightrule=0pt,
      fontupper=\color{colorChapter}\sffamily\bfseries\huge,
      width=\textwidth-3.2mm,  % account for shadow width.
      top=0.6cm, 
      bottom=0.6cm
    ]
    \MakeUppercase{#1}
    \end{tcolorbox}%
  }
  
\end{comment}

